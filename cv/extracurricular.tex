%-------------------------------------------------------------------------------
%	SECTION TITLE
%-------------------------------------------------------------------------------
\cvsection{Extracurricular Activity}


%-------------------------------------------------------------------------------
%	CONTENT
%-------------------------------------------------------------------------------
\begin{cventries}

%---------------------------------------------------------
\cventry
	{Developer and main learner of the day}
	{Embedded Projects}
	{home, Kraków, Poland}
	{Dec. 2017 - PRESENT}
	{
		\begin{cvitems} % Description(s) of experience/contributions/knowledge
			\item {In process of creation of a series of controllers to monitor the weather conditions in and out the house, listen to orders sent by GSM, keep track of scheduled events. This has been the project around which many things have been revolving.}
			\item {When facing memory size limitations, I had to delve into bitwise operation, compiler flags, and runtime calling conventions.}
			\item {When facing the implementation of a screen view for the system, I got involved into UI design patterns.}
	        \item {When facing timing responsiveness issues, I had to discover the concept of hardware interrupts, only to later discover the big world of RTOS, where I swim now.}
			\item {Every new challenge means an opportunity to devour a new book, which that alone has make this project very fruitful.}
		\end{cvitems}
	}

%---------------------------------------------------------
  \cventry
    {Mentor Staff}
    {Programming Languages}
    {Coursera}
    {Jan 2018 - PRESENT}
    {
      \begin{cvitems} % Description(s) of experience/contributions/knowledge
        \item {After having completed, \href{https://www.coursera.org/learn/programming-languages}{this course on Abstract Programming Languages} with the highest marks, I was selected to mentor.}
        \item {Studied Type Systems, Closures, Abstract Syntax Trees, Language implementations, Type Safety, Introduction to $\lambda$-Calculus, a comparison between the Object-Oriented Paradigm against the Functional one.}
        \item {Implemented a REPL in a Lisp language for a hypothetical language with closures.}
        \item {Implemented a Type Inference machine for Standard ML, in Standard ML.}
        \item {Implemented a Tetris game in Ruby using Object-Oriented design.}
        \item {Giving support to new students, participating in the forums and writing articles to promote curiosity and participation, and solving their doubts.}
      \end{cvitems}
    }

%---------------------------------------------------------
  \cventry
    {Student}
    {CS50 by Harvard}
    {edX}
	{Mar. 2017 - May. 2017} % Date(s)
    {
		\begin{cvitems} % Description(s) of experience/contributions/knowledge
			\item {Perhaps the official beginning of the Computer Sciences quest. By one of the best Computer Sciences instructors out there.}
			\item {Introduction to algorithms, data abstractions, machine architecture.}
		\end{cvitems}
    }

%---------------------------------------------------------
\cventry
	{Student}
	{Other Materials}
	{home, Kraków, Poland}
	{May 2017 - PRESENT}
	{
		\begin{cvitems} % Description(s) of experience/contributions/knowledge
			\item {Constantly checking materials Universities are publishing for free.}
			\item {Languages: semantics, lambda calculus, compilers (Delhi and Washington University materials).}
			\item {Algorithms and Data Abstractions (Stanford's CS106B).}
			\item {Computer Architecture: Von Neumann and Harvard Architecture, the almighty Stack (Jerusalem University).}
			\item {Software Design: Design Patterns, Test-Driven Development (by involvement into \href{https://github.com/rubberduck-vba/Rubberduck}{Rubberduck's VBA project on GitHub).}}
		\end{cvitems}
	}

%---------------------------------------------------------
\end{cventries}
