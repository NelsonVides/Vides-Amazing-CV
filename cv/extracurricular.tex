%-------------------------------------------------------------------------------
%	SECTION TITLE
%-------------------------------------------------------------------------------
\cvsection{Extracurricular Activity}

%-------------------------------------------------------------------------------
%	CONTENT
%-------------------------------------------------------------------------------
\begin{cventries}

    %---------------------------------------------------------
    \cventry{Mentor Staff in \href{https://www.coursera.org/learn/programming-languages}{\bulurl{Coursera}}}
    {Programming Languages Theory}
    {Coursera}
    {Jan 2018 --- PRESENT}
    {
        \begin{cvitems} % Description(s) of experience/contributions/knowledge
        \item {Studied Type Systems, Closures, Abstract Syntax Trees, Language implementations, Type
                Safety, $\lambda$-Calculus, and an analysis between the Object-Oriented Paradigm
                vs.\ the Functional Paradigm. Implemented a REPL in a Racket for a theoretical
                simply typed language with closures; a Type Inference machine for Standard ML, in
                Standard ML;\@ and a Tetris game in Ruby exploiting Object-Oriented ideas.}
        \item {Currently giving support to new students, participating in the forums and writing articles to promote curiosity and participation.}
        \end{cvitems}
    }

    %---------------------------------------------------------
    \cventry{Student}
    {Other Materials}
    {home, Kraków, Poland}
    {-}
    {
        \begin{cvitems} % Description(s) of experience/contributions/knowledge
        \item {Open Source Contributions: \href{https://github.com/rubberduck-vba/Rubberduck}{\bulurl{Rubberduck}'s VBA project on GitHub}: Design Patterns, Language Analysis, Test-Driven Development, CD\&CI.}
        \item {Daily reader of ACM research, with an special interest in Programming Languages Theory.}
        \item {Daily endeavour to Type Theory studies, Language Semantics, $\lambda$-Calculus and other calculi, Category Theory.}
        \end{cvitems}
    }

    %---------------------------------------------------------
\end{cventries}
